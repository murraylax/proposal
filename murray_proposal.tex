\documentclass{beamer}
\usepackage{beamerthemeshadow}
%\usepackage{beamercolorthemedolphin}
\usepackage{lastpage}

\usepackage{xcolor}
\usepackage{pgf}

\newcommand{\bi}{\begin{itemize}}
\newcommand{\ei}{\end{itemize}}
\newcommand{\be}{\begin{enumerate}}
\newcommand{\ee}{\end{enumerate}}
\newcommand{\bd}{\begin{description}}
\newcommand{\ed}{\end{description}}
\newcommand{\prbf}[1]{\textbf{#1}}
\newcommand{\prit}[1]{\textit{#1}}
\newcommand{\beq}{\begin{equation}}
\newcommand{\eeq}{\end{equation}}
\newcommand{\bdm}{\begin{displaymath}}
\newcommand{\edm}{\end{displaymath}}

\newcommand{\ft}[1]{
  \frametitle{\begin{tabular}{p{4in}r} \textcolor{white}{#1} & \small{\textcolor{white}{\thepage$~$ / \pageref{LastPage}}}\end{tabular}}
  \setbeamercovered{transparent=18}
}

\newcommand{\stepinv}{\setbeamercovered{invisible}}
\newcommand{\stopinv}{\setbeamercovered{transparent=18}}
\newcommand{\uncoverinv}[1]
{
  \setbeamercovered{invisible}
  \uncover<+->{#1}
  \setbeamercovered{transparent=18}
}
\newcommand{\ans}[1]{\textcolor{blue}{#1}}
\newcommand{\ansinv}[1]
{
  \setbeamercovered{invisible}
  \uncover<+->{\textcolor{blue}{#1}}
  \setbeamercovered{transparent=18}
}
\newcommand{\setinv}{\setbeamercovered{invisible}}
\newcommand{\setvis}{\setbeamercovered{transparent=18}}
\newcommand{\centerpic}[2]
{
  \begin{center}
  \includegraphics[#1]{#2}
  \end{center}
}
\newcommand{\h}[1]{\hat{#1}}
\newcommand{\ds}{\displaystyle}

\definecolor{mycolor}{rgb}{0.125,0.5,0.05}
\usecolortheme[named=mycolor]{structure}

\title{Three Essays in Adaptive Expectations in New Keynesian Monetary Economies}
\author[Dissertation Proposal. Indiana University. May 2007.]{James Murray}
\date{May 11, 2007}

\begin{document}

\frame{\titlepage}
\setcounter{page}{1}

\frame{
  \ft{Three Essays}
  \be
  \item Empirical Significance of Learning in a New Keynesian Model with Firm-Specific Capital
  \item Regime Switching, Learning, and the Great Moderation
  \item An Empirical Examination of Alternative Expectations Frameworks
  \ee
}

\frame{
  \ft{Paper 1: Learning with Firm-Specific Capital}
  \bi
  \item Purpose: Determine what features of U.S. data (if any) learning can explain.
  \item Learning has been suggested to deliver features:
    \bi
    \item Orphanides and Williams (2005): ``Inflation scares''.
    \item Milani (2005): Persistence in output and inflation.
    \item Volatility persistence of inflation.
    \item Primiceri (2005): Great Moderation.
    \ei
  \item Estimate a NK model with (constant gain) learning and RE by MLE.
  \item Examine forecast errors, evolution of shocks, and evolution of expectations.
  \ei
}

\frame
{
  \ft{NK Model: Consumers}
  \textbf{Utility function:}
  \bdm U_0 = E_0^* \sum_{t=0}^{\infty} \beta^t \left[ \frac{1}{1-\sigma} \xi_t \left(c_t(i) - \eta c_{t-1}(i)\right)^{1-\sigma} - \frac{1}{1+\mu} n_t(i)^{1+\mu} \right] \edm
  \bi
  \item $E_t^*$: possibly non-rational expectations operator.
  \item $c_t(i)$: consumption at time $t$. 
  \item $n_t(i)$: labor supply at time $t$.
  \item $\xi_t$: common preference shock.
  \item $\beta$: discount factor.
  \item $\sigma \in (0,\infty)$: related to the intertemporal elasticity of substitution.
  \item $\eta \in [0,1)$: degree of habit formation.
  \ei
}

\frame
{
  \ft{NK Model: Production}
  \textbf{Final good production:}
  \bdm y_t = \left[ \int_0^1 y_t(i)^{\frac{\theta-1}{\theta}} di \right]^{\frac{\theta}{\theta-1}} \edm
  \bi
  \item $y_t$ output of final good, $y_t(i)$ output of intermediate good $i$.
  \item $\theta \in (1,\infty)$: elasticity of substitution in production.
  \ei
  \textbf{Intermediate goods production:}
  \bdm y_t(i) = z_t k_t(i)^{\alpha} n_t(i)^{1-\alpha} \edm
  \bi
  \item $z_t$: common technology shock.
  \item $k_t(i)$: firm-specific capital good.
  \ei
}

\frame
{
  \ft{NK Model: Firm-Specific Investment}
  \bi
  \item Final good is converted to a firm-specific capital good.
  \item Investment of $I_t(i)$ leads to capital stock next period:
  \ei
  \bdm k_{t+1}(i) = (1-\delta) k_t(i) + \mu_t I_t(i) - \frac{\phi}{2} \left[\frac{k_{t+1}(i)}{k_t(i)} - 1 \right]^2 k_t(i) \edm
  \bi
  \item $\mu_t$: common investment technology shock.
  \item $\delta$: depreciation rate.
  \item $\phi$: capital adjustment cost parameter.
  \ei
}

\frame
{
  \ft{NK Model: Sticky prices}
  \bi
  \item Follow Calvo (1983) pricing: fraction $1-\omega$ firms re-optimize their price each period.
  \item Inflation indexation: Those who cannot re-optimize may adjust according to:
    \bdm p_{t}(i) = p_{t-1}(i) + \gamma \pi_{t-1} \edm
  \item Even with endogenous capital (Woodford, 2005), leads to Phillips curve:
  \bdm \pi_t = \frac{\gamma}{1+\beta\gamma} \pi_{t-1} +  \frac{\beta}{1+\beta\gamma} E_t^* \pi_{t+1} + \frac{(1-\omega)(1-\omega\beta)}{\nu \omega (1+\beta \gamma)} \h{s}_t \edm
  \item $\h{s}_t$: average marginal cost in the economy (percentage deviation from steady state).
  \item $\nu$: function of many parameters (no closed form).
  \ei
}

\frame
{
  \ft{NK Model: Wrap up}
  \bi
  \item Monetary policy: $\h{r}_t = \rho_r \h{r}_{t-1} + (1-\rho_r) \left(\psi_{\pi} \pi_t + \psi_y \h{y}_t \right) + \epsilon_{r,t}$
    \bi
    \item $\psi_{\pi} \in (0,\infty)$: feedback on inflation.
    \item $\psi_{y} \in (0,\infty)$: feedback on output.
    \item $\rho_r \in (0,1)$: smoothing parameter.
    \ei
  \item Non-policy shocks (percentage deviations from steady state) are AR(1):
    \bi
    \item Preference shock: $\h{\xi}_t = \rho_{\xi} \h{\xi}_{t-1} + \epsilon_{\xi,t}$ 
    \item Technology shock: $\h{z}_t = \rho_{z} \h{z}_{t-1} + \epsilon_{z,t}$ 
    \item Investment shock: $\h{\mu}_t = \rho_{\mu} \h{\mu}_{t-1} + \epsilon_{\mu,t}$ 
    \ei
  \item Market clearing condition: $\ds y_t = c_t + I_t$
  \ei
}

\frame
{
  \ft{Learning in DSGEs}
  Suppose a DSGE model of the form:
  \bdm \Omega_{0} x_t = \Omega_{1} x_{t-1} + \Omega_{2} E_t^* x_{t+1} + \Psi \epsilon_t \edm
  \vspace{-1pc}
  \bi
  \item $x_t$ vector of time $t$ variables, all observable to agents.
  \item $E_t^*$: possibly non-rational expectations operator.
  \ei
  \vspace{1pc}
  Rational expectations solution implies:
  \bdm E_t x_{t+1} = G x_t \edm
  \vspace{-1pc}
  \bi
  \item Agents know the form of this solution, but estimate elements of $G$ by least squares.
  \item Use as explanatory variables past observations of $x_t^k$.
  \item $x_t^k$ includes a constant and some or all variables in $x_t$.
  \ei
}
  
\frame
{
  \ft{Ordinary Least Squares}
  \bi 
  \item Let $G_t^k$ include non-zero columns of $G$ and a constant.
  \item Ordinary least squares estimate for $G^k$ at time $t$:
    \bdm \left(\hat{G}_t^k\right)' = \left( \frac{1}{t-1} \sum_{\tau=1}^{t-1} x_{\tau-1}^k {x_{\tau-1}^{k}}' \right)^{-1} \left( \frac{1}{t-1} \sum_{\tau=1}^{t-1} x_{\tau-1}^k x_{\tau}' \right) \edm
  \item Least squares forecast:
    \beq \label{eq:agfore} E_t^* x_{t+1} = \h{g}_{0,t} + \h{G}_t E_t^* x_t = (I + \h{G}_t)\h{g}_{0,t} + \h{G}_t^2 x_{t-1} \eeq
  \item Evolution of $\h{G}_t^k$ in recursive form:
    \beq \label{eq:lnG} \hat{G}_t^k = \hat{G}_{t-1}^k + g_t (x_{t-1} - \hat{G}_{t-1}^k x_{t-2}^k) {x_{t-2}^k}' R_t^{-1} \eeq
    \beq \label{eq:lnX} R_t = R_{t-1} + g_t (x_{t-2}^k {x_{t-2}^k}' - R_{t-1}) \eeq
  \item where $g_t=1/(t-1)$ is the learning gain.
  \ei
}

\frame
{
  \ft{Constant Gain Learning}
  \bi
  \item Ordinary least squares $\rightarrow$ learning dynamics disappear.
  \item Constant gain:
    \bi 
    \item Assumes $g$ is constant.
    \item Dynamics of expectations depend on the size of the learning gain.
    \item Learning dynamics persist in the long run.
    \item Appropriate (MSV solution) initial condition + $(g=0)$ $\rightarrow$ RE.
    \ei
  \ei
}


\frame
{
  \ft{Initializing Learning Algorithm}
  \bi
  \item $\hat{G}_t$ and $R_t$ must be initialized for estimation.
  \item This paper: MSV solution, Expected Variance of state vector under RE.
    \bi
    \item Problem: Learning dynamics smallest when near RE solution.
    \ei
  \item More popular: Use a VAR(1) from pre-sample data
    \bi
    \item Problem: initial condition far away from long run steady state.
    \item Problem: latent variables - capital stock, structural shocks.
    \ei
  \item Ad hock assumptions.  In Milani (2005) coefficients on past inflation set to zero.
  \item Joint estimation:  Zha (Lecture notes, 2005).
    \bi
    \item Problems: Possibly many additional parameters, possible over-fitting problem.  
    \ei
  \ei
}

\frame
{
  \ft{Estimation Procedure}
  \bi
  \item Estimate by MLE using Kalman filter.
  \item Data: Quarterly data from 1957 through 2005.
    \bi
    \item Output growth: growth rate of real GDP.
    \item Investment growth: growth rate of real aggregate domestic investment.
    \item Inflation: growth rate of GDP deflator.
    \item Interest rate: federal funds rate.
    \ei
  \ei
}

\frame
{
\ft{Results with No Capital Accumulation}
\begin{table}[ht]
\begin{scriptsize}
\begin{center}
\begin{tabular}{|l|c|r@{.}l|r@{.}l|} \hline
Description & Parameter & \multicolumn{2}{|c|}{Learning} & \multicolumn{2}{|c|}{RE} \\ \hline
Learning gain & $g$ & 0&017225*** & \multicolumn{2}{|c|}{--} \\ 
Discount factor & $\beta$ & 0&994528*** & 0&994893*** \\ 
Habit formation & $\eta$ & 0&269967** & 0&213315 \\ 
Inverse elasticity sub. & $\sigma$ & 16&264823 & 17&589033 \\ 
Elasticity of sub. production & $\theta$ & 11&459122 & 10&030577 \\ 
Inverse elasticity labor supply & $\mu$ & 2&509288 &  1&791766 \\ 
Calvo parameter & $\omega$ & 0&715756 & 0&713950 \\ 
Inflation indexation & $\gamma$ & 0&441247*** & 0&958006*** \\ 
MP interest rate smoothing & $\rho_r$ & 0&866560*** & 0&823025*** \\ 
MP feedback on output & $\psi_y$ & 0&123166** & 0&055319* \\ 
MP feedback on inflation & $\psi_{\pi}$ & 0&995049*** & 1&067211*** \\ 
Pref. shock persistence & $\rho_{\xi}$ & 0&931082*** & 0&962095***  \\ 
Tech. shock persistence & $\rho_{z}$ & 0&000010 & 0&000010 \\ 
Steady state inflation & $\pi^{*}$ & 2&814514*** & 3&296221** \\ 
Std. dev. technology shock & $\sigma_z$ & 0&301741* & 0&340167 \\ 
Std. dev. preference shock & $\sigma_{\xi}$ & 0&280402* & 0&254504 \\ 
Std. dev. interest rate shock & $\sigma_{r}$ & 0&002283*** & 0&002344*** \\ \hline
\end{tabular}

* Significantly different from zero at the 10\% level.\newline
** Significantly different from zero at the 5\% level.\newline
*** Significantly different from zero at the 1\% level.\newline
\end{center}
\end{scriptsize}
\end{table}
}


\frame
{
  \ft{Results with No Capital Accumulation}
  \bi
  \item Learning statistically significant.
  \item Learning leads to lower degree of inflation indexation.
  \item Habit formation still significant source of persistence.
  \item Very similar estimates for the degree of price flexibility.
  \item Possible reasons for differences with Milani (2005):
    \bi
    \item MLE vs. Bayesian methods.
    \item Different initial condition for recursive learning process.
    \item Different data: Growth rate of real GDP vs. CBO measure of output gap.
    \ei
  \ei
}

\frame
{
  \ft{In-Sample One Quarter Ahead Forecast Errors}
  \begin{center}
  \vspace*{-0.1in}\hspace*{-0.24in}\begin{tabular}{ccc}
  \multicolumn{3}{c}{\textbf{Learning}}  \\
  \small{Output} & \small{Inflation} & \small{Interest Rate} \\
  \includegraphics[scale=0.23]{plots/ln_nocap_full_res_yfe.png} & \includegraphics[scale=0.23]{plots/ln_nocap_full_res_pife.png} & \includegraphics[scale=0.23]{plots/ln_nocap_full_res_rfe.png} \\ \\
  \multicolumn{3}{c}{\textbf{Rational Expectations}}  \\
  \small{Output} & \small{Inflation} & \small{Interest Rate} \\
  \includegraphics[scale=0.23]{plots/re_nocap_full_res_yfe.png} & \includegraphics[scale=0.23]{plots/re_nocap_full_res_pife.png} & \includegraphics[scale=0.23]{plots/re_nocap_full_res_rfe.png} \\ \\ \\
  \end{tabular}
  \end{center}
}

\frame
{
  \ft{In-Sample One Quarter Ahead Forecast Errors}
  \bi
  \item Forecast errors are very similar for Learning and RE.
  \item Forecast errors are more volatile in 1970s.
  \item Huge forecast errors for federal funds rate in late 70s, early 80s.
  \item Learning appears not to be explaining U.S. dynamics better than RE.
  \item Both fail to explain high volatility in 70s with low volatility after mid 80s.
  \ei
}

\frame
{
  \ft{Eight Quarter Out-of-Sample Forecasts}
  \bi
  \item Estimate the model through 1989:Q4.
  \item Use estimated parameters to forecast 1990:Q1 - 2005:Q4.
  \item For each quarter, forecast eight periods ahead.
  \item Given forecasts of each horizon, compute MSE.
  \item Do the same for a VAR(4) and Litterman (1986) BVAR(4).
  \ei
}

\frame
{
  \ft{Eight Quarter Out-of-Sample Forecasts}
  \begin{center}
  \begin{tabular}{ccc}
  Output & Inflation  \\ 
  \includegraphics[scale=0.3]{plots/mse_nocap_y.png} & \includegraphics[scale=0.3]{plots/mse_nocap_pi.png} \\
  Interest Rate & \\
  \includegraphics[scale=0.3]{plots/mse_nocap_r.png} & \\
  \end{tabular}
  \end{center}
}

\frame
{
  \ft{Smoothed Technology Shocks}
  \begin{center}
  \vspace*{-0.1in}\hspace*{-0.24in}\begin{tabular}{ccc}
  \multicolumn{3}{c}{\textbf{Learning}}  \\
  \small{Technology Shock} & \small{Preference Shock} & \small{Interest Rate Shock} \\
  \includegraphics[scale=0.23]{plots/ln_nocap_full_res_techsh.png} & \includegraphics[scale=0.23]{plots/ln_nocap_full_res_prefsh.png} & \includegraphics[scale=0.23]{plots/ln_nocap_full_res_ffsh.png} \\ \\
  \multicolumn{3}{c}{\textbf{Rational Expectations}}  \\
  \small{Technology Shock} & \small{Preference Shock} & \small{Interest Rate Shock} \\
  \includegraphics[scale=0.23]{plots/re_nocap_full_res_techsh.png} & \includegraphics[scale=0.23]{plots/re_nocap_full_res_prefsh.png} & \includegraphics[scale=0.23]{plots/re_nocap_full_res_ffsh.png} \\ \\ \\
  \end{tabular}
  \end{center}
}

\frame
{
  \ft{Agents' Expectations}
  \begin{center}
  \vspace*{-0.1in}\hspace*{-0.24in}\begin{tabular}{ccc}
  \multicolumn{3}{c}{\textbf{Learning}}  \\
  \small{Consumption} & \small{Inflation} & \small{Preference shock} \\
  \includegraphics[scale=0.23]{plots/ln_nocap_full_res_cag.png} & \includegraphics[scale=0.23]{plots/ln_nocap_full_res_piag.png} & \includegraphics[scale=0.23]{plots/ln_nocap_full_res_xiag.png} \\ \\
  \multicolumn{3}{c}{\textbf{Rational Expectations}}  \\
  \small{Consumption} & \small{Inflation} & \small{Preference shock} \\
  \includegraphics[scale=0.23]{plots/re_nocap_full_res_cag.png} & \includegraphics[scale=0.23]{plots/re_nocap_full_res_piag.png} & \includegraphics[scale=0.23]{plots/re_nocap_full_res_xiag.png} \\ \\ \\
  \end{tabular}
  \end{center}
}

\frame
{
\ft{Results with Endogenous Capital Accumulation}
\begin{table}[ht]
\begin{scriptsize}
\begin{center}
\begin{tabular}{|l|c|r@{.}l|r@{.}l|} \hline
Description & Parameter & \multicolumn{2}{|c|}{Learning} & \multicolumn{2}{|c|}{RE} \\ \hline
Learning gain & $g$ & 0&024327** & \multicolumn{2}{|c|}{--} \\ 
Discount factor & $\beta$ & 0&993366*** & 0&992763*** \\ 
Habit formation & $\eta$ & 0&281563** & 0&286030* \\ 
Inverse elasticity sub. & $\sigma$ & 18&558950 & 16&328695 \\ 
Elasticity of sub. production & $\theta$ & 13&122359 & 7&317025 \\ 
Inverse elasticity labor supply & $\mu$ & 3&329474 & 6&219549 \\ 
Capital share of income & $\alpha$ & 0&174750 & 0&186636 \\ 
Depreciation rate & $\delta$ & 0&163489 & 0&299872 \\ 
Cost of adjusting capital & $\phi$ & 13&455016 & 11&558055 \\ 
Calvo parameter & $\omega$ & 0&658099 & 0&774919 \\ 
Inflation indexation & $\gamma$ & 0&404221*** & 0&760702*** \\ 
MP interest rate smoothing & $\rho_r$ & 0&869496*** & 0&859279*** \\ 
MP feedback on output & $\psi_y$ & 0&064696* & 0&128857*** \\ 
MP feedback on inflation & $\psi_{\pi}$ & 0&992672*** & 0&967512*** \\ 
Pref. shock persistence & $\rho_{\xi}$ & 0&984689*** & 0&980813***  \\ 
Tech. shock persistence & $\rho_{z}$ & 0&012960 & 0&000010 \\ 
Inv. shock persistence & $\rho_{\mu}$ & 0&804935*** & 0&824007*** \\ 
Steady state inflation & $\pi^{*}$ & 3&570552*** & 4&073905*** \\ 
Std. dev. technology shock & $\sigma_z$ & 0&229175** & 0&400000** \\ 
Std. dev. investment shock & $\sigma_{\mu}$ & 0&060246*** & 0&052502***  \\ 
Std. dev. preference shock & $\sigma_{\xi}$ & 0&231848* & 0&214337* \\ 
Std. dev. interest rate shock & $\sigma_{r}$ & 0&002291*** & 0&002286*** \\ \hline
\end{tabular}
\end{center}
\end{scriptsize}
\end{table}
}

\frame
{
  \ft{Results with Endogenous Capital Accumulation}
  \bi
  \item Learning remains statistically significant.
  \item Including capital leads to a lower degree of inflation indexation.
  \item Learning further lowers the degree of inflation indexation.
  \item Habit formation still significant source of persistence.
  \item Learning leads to lower estimates for the degree of price flexibility.
  \item Learning leads to lower estimates for the variance of the technology shock.
  \ei
}

\frame
{
  \ft{One Quarter Ahead Forecast Errors}
  \begin{center}
  \hspace*{-0.32in}\begin{tabular}{cccc}
  \multicolumn{4}{c}{\textbf{Learning}}  \\
  \small{Output} & \small{Investment} & \small{Inflation} & \small{Interest Rate} \\
  \includegraphics[scale=0.18]{plots/ln_cap_full_res_yfe.png} & \includegraphics[scale=0.18]{plots/ln_cap_full_res_Ife.png} & \includegraphics[scale=0.18]{plots/ln_cap_full_res_pife.png} & \includegraphics[scale=0.18]{plots/ln_cap_full_res_rfe.png} \\ \\
  \multicolumn{4}{c}{\textbf{Rational Expectations}}  \\
  \small{Output} & \small{Investment} & \small{Inflation} & \small{Interest Rate} \\
  \includegraphics[scale=0.18]{plots/re_cap_full_res_yfe.png} & \includegraphics[scale=0.18]{plots/re_cap_full_res_Ife.png} & \includegraphics[scale=0.18]{plots/re_cap_full_res_pife.png} & \includegraphics[scale=0.18]{plots/re_cap_full_res_rfe.png} \\ \\ \\
  \end{tabular}
  \end{center}
}

\frame
{
  \ft{Eight Quarter Out-of-Sample Forecasts}
  \begin{center}
  \begin{tabular}{ccc}
  Output & Inflation  \\ 
  \includegraphics[scale=0.3]{plots/mse_cap_y.png} & \includegraphics[scale=0.3]{plots/mse_cap_pi.png} \\
  Investment & Interest Rate \\
  \includegraphics[scale=0.3]{plots/mse_cap_I.png} & \includegraphics[scale=0.3]{plots/mse_cap_r.png} \\
  \end{tabular}
  \end{center}
}

\frame
{
  \ft{Smoothed Technology Shocks}
  \begin{center}
  \hspace*{-0.32in}\begin{tabular}{cccc}
  \multicolumn{4}{c}{\textbf{Learning}}  \\
  \small{Technology Shock} & \small{Investment Shock} & \small{Preference Shock} & \small{Interest Rate Shock} \\
  \includegraphics[scale=0.18]{plots/ln_cap_full_res_techsh.png} & \includegraphics[scale=0.18]{plots/ln_cap_full_res_invsh.png} & \includegraphics[scale=0.18]{plots/ln_cap_full_res_prefsh.png} & \includegraphics[scale=0.18]{plots/ln_cap_full_res_ffsh.png} \\ \\
  \multicolumn{4}{c}{\textbf{Rational Expectations}}  \\
  \small{Technology Shock} & \small{Investment Shock} & \small{Preference Shock} & \small{Interest Rate Shock} \\
  \includegraphics[scale=0.18]{plots/re_cap_full_res_techsh.png} & \includegraphics[scale=0.18]{plots/re_cap_full_res_invsh.png} & \includegraphics[scale=0.18]{plots/re_cap_full_res_prefsh.png} & \includegraphics[scale=0.18]{plots/re_cap_full_res_ffsh.png} \\ \\ \\
  \end{tabular}
  \end{center}
}


\frame
{
  \ft{Agents' Expectations}
  \begin{center}
  \vspace*{-0.1in}\hspace*{-0.24in}\begin{tabular}{ccc}
  \multicolumn{3}{c}{\textbf{Learning}}  \\
  \small{Consumption} & \small{Inflation} & \small{Capital stock} \\
  \includegraphics[scale=0.23]{plots/ln_cap_full_res_cag.png} & \includegraphics[scale=0.23]{plots/ln_cap_full_res_piag.png} & \includegraphics[scale=0.23]{plots/ln_cap_full_res_kag.png} \\ \\
  \multicolumn{3}{c}{\textbf{Rational Expectations}}  \\
  \small{Consumption} & \small{Inflation} & \small{Capital stock} \\
  \includegraphics[scale=0.23]{plots/re_cap_full_res_cag.png} & \includegraphics[scale=0.23]{plots/re_cap_full_res_piag.png} & \includegraphics[scale=0.23]{plots/re_cap_full_res_kag.png} \\ \\ \\
  \end{tabular}
  \end{center}
}

\frame
{
  \ft{Conclusion}
  \bi
  \item Learning (small) successes:
    \bi
    \item Lower estimates for inflation indexation, Calvo parameter (using capital).
    \item Some evidence of inflation scares.
    \item Better performance out-of-sample, especially for inflation.
    \ei
  \item Learning failures:
    \bi
    \item Very similar to RE in explaining the data.
    \item 1970s inflation, monetary policy.
    \ei
  \item Hypotheses:
    \be
    \item Sensitivity to initial conditions suggests learning explains better dynamics following structural change.
    \item Dynamic gain may deliver larger effects.
    \item Williams (2003), Primiceri (2005): Learning about structural features may deliver quantitatively larger effects.
    \ee
  \ei
}

\frame
{
  \ft{Paper 2: Switching, Learning, and Moderation}
  \bi
  \item Addresses my first two hypotheses. 
  \item ``Bad luck'' regime changes (not my language).
    \bi
    \item Variances of non-policy shock switch between high/low states according to Markov chain.
    \item Sims and Zha (2006): evidence of Great Moderation points towards this type of switching.
    \item Bullard and Singh (2007): bad luck + Bayesian learning can explain Great Moderation.
    \ei
  \item Learning framework: Marcet and Nicolini (2003) endogenous learning gain.
  \ei
}

\frame
{
  \ft{Endogenous Learning Gain}
  \bi
  \item When recent forecast errors get big, agents switch to higher learning gain (higher discount for past data).
  \item Let $\alpha_t \equiv 1/g_t$.  In OLS case $\alpha_t$ is sample size.
  \beq \label{eq:mngain} \alpha_t = \left\{ \begin{array}{cl} \ds \alpha_{t-1} + 1 & \ds \mbox{     if  } \frac{1}{J} \sum_{j=1}^{J} \frac{1}{n} \sum_{v=1}^{n} \left| \frac{x_{t-j}(v) - \hat{G}_{t-j}^*(v) x_{t-j-1}^*}{\hat{G}_{t-j}^*(v) x_{t-j-1}^*} \right| < \nu \\ \ds \alpha & \ds \mbox{     otherwise} \end{array} \right \eeq
  \item Notation:
    \bi
    \item $n$ is the number of variables in $x_t$.
    \item $x_{t-j}(v)$: $v$th element of $x_{t-j}$.
    \item $\hat{G}_{t-j}^*(v)$: $v$th row of $\hat{G}_{t-j}^*(v)$.
    \item $\alpha \equiv 1/g$ is the constant gain.
    \item $\nu \in (0,\infty)$ is a threshold level.
    \ei
  \ei
}

\frame
{
  \ft{Regime Switching Shocks}
  \bi
  \item Serially correlated shocks:
  \beq z_t = \rho_z z_{t-1} + \epsilon_{z,t}(s_t) \eeq
  \beq \xi_t = \rho_{\xi} \xi_{t-1} + \epsilon_{\xi,t}(s_t) \eeq
  \item Innovations for a given regime are mean zero and iid.
  \beq \label{eq:vars}
Var\left( \epsilon_{i,t}(s_t) \right) = \left\{ \begin{array}{c} \sigma_{i,1}^2 \mbox{,    if $s_t=1$} \\ \sigma_{i,1}^2 \mbox{,    if $s_t=2$} \end{array} \right \eeq
  \item Where $i=\xi,z,\mu$ (preference, technology, and investment shocks).
  \item Transition matrix:
    \beq \label{eq:tran} P = \left[ \begin{array}{cc} p_1 & 1-p_1 \\ 1-p_2 & p_2 \end{array} \right]. \eeq
  \ei
}

\frame
{
  \ft{Estimation}
  \bi
  \item Kim (1994), Kim and Nelson (1999): form the likelihood.
    \bi
    \item Kalman filter for state-space models.
    \item Hamilton (1989) filter for regime switching.
    \ei
  \item Plan to estimate:
    \bi
    \item Point estimates of high and low variances: $\sigma_{z,1}^2, \sigma_{\xi,1}^2$, and $\sigma_{z,1}^2, \sigma_{\xi,1}^2$.
    \item Learning gain parameters:  $g$ and $\nu$.
    \item Smoothed estimates of the regime probabilities for each period.
    \item Smoothed estimates of the shocks.
    \item Out-of-sample forecasts of the model.
    \ei
  \ei
}

\frame
{
  \ft{Interesting questions}
  \bi
  \item Can one identify regime switches and dynamic learning gain changes simultaneously?
  \item If so, does learning play a larger role with regime changes?
  \item With endogenous gain learning, how sizable of an increase in shock volatility is necessary to explain volatile periods such as the 1970s.
  \ei
}

\frame
{
  \ft{Paper 3: Alternative Expectations Frameworks}
  \bi
  \item Deviating from rational expectations is becoming more popular.
  \item Alternative frameworks even in least squares learning:
    \bi
    \item Decreasing, constant, dynamic gain.
    \item Structural learning.
    \item Preston (2005): Long run horizon learning.
    \item Branch, Carlson, Evans, and McGough (2006a, 2006b): Endogenous inattention.
    \ei
  \item Outside of least squares learning:
    \bi
    \item Bayesian learning.
    \item Rational inattention.
    \ei
  \ei
}

\frame
{
  \ft{Goal of this paper}
  \bi
  \item Estimate same NK model under alternative expectations frameworks.
  \item Examine forecast errors, smoothed shocks, agents expectations.
  \item Generate impulse response functions (IRF) on estimated model.
  \item Interesting questions:
    \bi
    \item Are some expectations frameworks easier to identify?
    \item How well to different expectations frameworks explain U.S. experience of changing volatility and changing persistence better?
    \item Does the expectations framework matter?  They are not mathematically equivalent; but are they observationally equivalent?
    \item Do IRF from each framework imply different dynamics?
    \ei
  \ei
}


\end{document}

